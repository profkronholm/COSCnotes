% !TEX TS-program = pdflatex
% !TEX encoding = UTF-8 Unicode

% This is a simple template for a LaTeX document using the "article" class.
% See "book", "report", "letter" for other types of document.

\documentclass[11pt, landscape]{article} % use larger type; default would be 10pt

\usepackage[utf8]{inputenc} % set input encoding (not needed with XeLaTeX)

%%% Examples of Article customizations
% These packages are optional, depending whether you want the features they provide.
% See the LaTeX Companion or other references for full information.

%%% PAGE DIMENSIONS
\usepackage[letterpaper, margin=1in]{geometry} % to change the page dimensions
%\geometry{a4paper} % or letterpaper (US) or a5paper or....
% \geometry{margins=2in} % for example, change the margins to 2 inches all round
% \geometry{landscape} % set up the page for landscape
%   read geometry.pdf for detailed page layout information

\usepackage{graphicx} % support the \includegraphics command and options


\usepackage{xcolor} % used for background color for code
\usepackage{menukeys} % for \cmd that looks like ⌘

% \usepackage[parfill]{parskip} % Activate to begin paragraphs with an empty line rather than an indent

%%% PACKAGES
\usepackage{booktabs} % for much better looking tables
\usepackage{array} % for better arrays (eg matrices) in maths
\usepackage{paralist} % very flexible & customisable lists (eg. enumerate/itemize, etc.)
\usepackage{verbatim} % adds environment for commenting out blocks of text & for better verbatim
\usepackage{subfig} % make it possible to include more than one captioned figure/table in a single float
% These packages are all incorporated in the memoir class to one degree or another...
\usepackage{listings} % for writing code blocks



%%% HEADERS & FOOTERS
\usepackage{fancyhdr} % This should be set AFTER setting up the page geometry
\pagestyle{fancy} % options: empty , plain , fancy
\renewcommand{\headrulewidth}{0pt} % customise the layout...
\lhead{}\chead{}\rhead{}
\lfoot{}\cfoot{\thepage}\rfoot{}

%%% SECTION TITLE APPEARANCE
\usepackage{sectsty}
\allsectionsfont{\rmfamily\upshape} % (See the fntguide.pdf for font help)
%\setcounter{section}{-1}
% (This matches ConTeXt defaults)

%%% ToC (table of contents) APPEARANCE
\usepackage[nottoc,notlof,notlot]{tocbibind} % Put the bibliography in the ToC
\usepackage[titles,subfigure]{tocloft} % Alter the style of the Table of Contents
\renewcommand{\cftsecfont}{\rmfamily\mdseries\upshape}
\renewcommand{\cftsecpagefont}{\rmfamily\mdseries\upshape} % No bold!

\usepackage{hyperref}

\definecolor{codegray}{gray}{0.9}
\newcommand{\code}[1]{\colorbox{codegray}{\texttt{#1}}}
\newcommand{\keyboard}[1]{{\sc{#1}}}
%%% END Article customizations



%%% The "real" document content comes below...

\title{Common Useful Terminal Commands}
\author{Bill Kronholm}
\date{} % Activate to display a given date or no date (if empty),
         % otherwise the current date is printed 

\begin{document}
\maketitle
\renewcommand{\arraystretch}{1.4}
\begin{tabular}{|l|l|l|l|}
\hline
Command to\dots & Linux & Windows & Mac OS X \\
\hline
change directory & \code{cd /path/to/directory/} & \code{cd \textbackslash path\textbackslash to\textbackslash directory\textbackslash}  & \code{cd /path/to/directory/} \\
\hline
change to home directory & \code{cd $\sim$} & \code{cd \%HOMEPATH\%} & \code{cd $\sim$} \\
\hline
navigate one directory up & \code{cd ..} & \code{cd ..} & \code{cd ..} \\
\hline
change to previous directory & \code{cd -} & not implemented natively & \code{cd -}\\
\hline
list directory contents & \code{ls} & \code{dir} & \code{ls} \\
\hline
create a copy of \code{file1} called \code{file2} & \code{cp file1 file2} & \code{copy file1 file2} & \code{cp file1 file2}\\
\hline
rename \code{file1} to \code{file2} & \code{mv file1 file2} & \code{rename file1 file2} & \code{mv file1 file2}\\
\hline
delete \code{file1} & \code{rm file1} & \code{del file1} & \code{rm file1} \\
\hline
create a zip archive of the directory \code{mydir} & \code{zip -r mydir.zip mydir} & not implemented natively & \code{zip -r mydir.zip mydir}\\
\hline
start the Python interpreter & \code{python} & \code{python} & \code{python} \\
\hline
execute the python script \code{mycode.py} & \code{python mycode.py} & \code{python mycode.py} & \code{python mycode.py}\\
\hline
print contents of file \code{file1} to screen & \code{cat file1} & \code{type file1}  & \code{cat file1}\\
\hline
clear the screen & \code{clear} & \code{cls} & \code{clear}\\
\hline
start gedit (assuming it is in \code{PATH}) & \code{gedit} & \code{gedit} & \code{gedit}\\
\hline
close the terminal window & \code{exit} & \code{exit} & \code{exit}\\
\hline
\end{tabular}
\end{document}
