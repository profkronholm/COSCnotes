% !TEX TS-program = pdflatex
% !TEX encoding = UTF-8 Unicode

% This is a simple template for a LaTeX document using the "article" class.
% See "book", "report", "letter" for other types of document.

\documentclass[11pt]{article} % use larger type; default would be 10pt

\usepackage[utf8]{inputenc} % set input encoding (not needed with XeLaTeX)

%%% Examples of Article customizations
% These packages are optional, depending whether you want the features they provide.
% See the LaTeX Companion or other references for full information.

%%% PAGE DIMENSIONS
\usepackage[letterpaper, margin=1in]{geometry} % to change the page dimensions
%\geometry{a4paper} % or letterpaper (US) or a5paper or....
% \geometry{margins=2in} % for example, change the margins to 2 inches all round
% \geometry{landscape} % set up the page for landscape
%   read geometry.pdf for detailed page layout information

\usepackage{graphicx} % support the \includegraphics command and options


\usepackage{xcolor} % used for background color for code
\usepackage{menukeys} % for \cmd that looks like ⌘

% \usepackage[parfill]{parskip} % Activate to begin paragraphs with an empty line rather than an indent

%%% PACKAGES
\usepackage{booktabs} % for much better looking tables
\usepackage{array} % for better arrays (eg matrices) in maths
\usepackage{paralist} % very flexible & customisable lists (eg. enumerate/itemize, etc.)
\usepackage{verbatim} % adds environment for commenting out blocks of text & for better verbatim
\usepackage{subfig} % make it possible to include more than one captioned figure/table in a single float
% These packages are all incorporated in the memoir class to one degree or another...
\usepackage{listings} % for writing code blocks



%%% HEADERS & FOOTERS
\usepackage{fancyhdr} % This should be set AFTER setting up the page geometry
\pagestyle{fancy} % options: empty , plain , fancy
\renewcommand{\headrulewidth}{0pt} % customise the layout...
\lhead{}\chead{}\rhead{}
\lfoot{}\cfoot{\thepage}\rfoot{}

%%% SECTION TITLE APPEARANCE
\usepackage{sectsty}
\allsectionsfont{\rmfamily\upshape} % (See the fntguide.pdf for font help)
%\setcounter{section}{-1}
% (This matches ConTeXt defaults)

%%% ToC (table of contents) APPEARANCE
\usepackage[nottoc,notlof,notlot]{tocbibind} % Put the bibliography in the ToC
\usepackage[titles,subfigure]{tocloft} % Alter the style of the Table of Contents
\renewcommand{\cftsecfont}{\rmfamily\mdseries\upshape}
\renewcommand{\cftsecpagefont}{\rmfamily\mdseries\upshape} % No bold!

\usepackage{hyperref}

\definecolor{codegray}{gray}{0.9}
\newcommand{\code}[1]{\colorbox{codegray}{\texttt{#1}}}
\newcommand{\keyboard}[1]{{\sc{#1}}}
%%% END Article customizations



%%% The "real" document content comes below...

\title{Setting up Python}
\author{Bill Kronholm}
\date{} % Activate to display a given date or no date (if empty),
         % otherwise the current date is printed 

\begin{document}
\maketitle

This guide will step you through installing Python 2.7 on your system as well as a text editor with nice highlighting features to make writing your programs a little easier.
The steps you need to take vary based on the system you are using.

\section{Accessing the terminal window}
\subsection{Linux} Press \keyboard{ctrl-alt-t} and a terminal window will open.
You should be in the home directory of the current user and see \code{username@machine:$\sim$\$~} or something similar.

Type \code{exit} and press \keyboard{Enter} to close the terminal window.

\subsection{Windows} There is no hot-key for getting the terminal window to open, so we will create one.
Hit the \keyboard{start} button and type \code{cmd}.
You should see the \code{cmd.exe} program appear at the top of the list.
Right-click on \code{cmd.exe} and select ``Pin to Start Menu.''
(If this is not an option, the it is already pinned to the start menu.)
Close the start menu.

Now reopen the start menu, right-click on ``Windows Command Processor,'' and select ``Properties.''
In the ``Start in'' field, enter the following: \code{\%HOMEPATH\%}.
Next, click in the ``Shortcut key'' field, and press \keyboard{ctrl-alt-t}.
Click ``Apply,'' then click ``OK.''

Now press \keyboard{ctrl-alt-t} and you should see a terminal window which displays \code{C:\textbackslash Users\textbackslash username>} or something similar.

Type \code{exit} and press \keyboard{Enter} to close the terminal window.

(Note: You could also hit the \keyboard{Start} key, type \code{cmd}, and hit \keyboard{Enter}, but this will not take you to your home directory by default.)

\subsection{Mac OS X}

Press \keyboard{\cmd-space} to open Spotlight, then hit \keyboard{Enter}.
A terminal window should appear and you should be in your home directory.

Type \code{exit} and press \keyboard{Enter} to close the terminal window.

\section{Installing Python 2.7}
\subsection{Linux}
Linux distributions ship with Python 2.7 already installed.
To check what version you are using, open a terminal window and type \code{python --version} and hit \keyboard{Enter}.
You should see it says \code{Python 2.7.X}.

\subsection{Windows}
Visit \href{https://www.python.org/downloads/release/python-2710/}{https://www.python.org/downloads/release/python-2710/} and choose the correct download link.
(It will be either be ``Windows x86-64 MSI installer'' if you are on a 64-bit machine, or else ``Windows x86 MSI installer'' if you are on a 32-bit machine.)
If prompted, select the option that allows you to save the file.
(But probably it will save automatically.)
Find the file and run it.
Follow the prompts to install for all users.
Select the default directory of \code{C:\textbackslash Python27\textbackslash}.
On the ``Customize'' page, scroll down to make sure that ``Add python.exe to Path'' will be installed.
If you are prompted for permission to install, allow the machine to do so.
Now just wait.

When it's done, open a terminal window and type \code{python --version} and hit \keyboard{Enter}.
You should get a response that says \code{Python 2.7.10}.

\subsection{Mac OS X}
Mac OS X comes with a version of python, but it's not what we want to use.
Instead, go to \href{https://www.python.org/downloads/release/python-2710/}{https://www.python.org/downloads/release/python-2710/}.
Click the ``Mac OS X 64-bit/32-bit installer'' and follow the prompts to install it.
(I can't help much more than this as I do not have a Mac to test this on.)

Once it is installed, open a terminal window and type \code{python --version} and hit \keyboard{Enter}.
You should get a response that says \code{Python 2.7.10}.


\section{Setting up a text editor}
The actual text editor you use is completely irrelevant and can be anything you'd like.
I recommend starting with gedit: \href{https://wiki.gnome.org/Apps/Gedit}{https://wiki.gnome.org/Apps/Gedit}.
Click ``Download'' and select the appropriate link for your operating system.
Choose the latest version and download the installation file.
Open it and follow the installation prompts on your screen.

Once gedit is installed, open it.
From the ``Edit'' menu, select ``Preferences.''
In the ``View'' tab, make sure ``Display line numbers'' is checked.
Check the box for ``Display right margin at column:'' and set the value to 79.
Switch to the ``Editor'' tab and check the box for ``Enable automatic indentation.''
Then click ``Close.''

You should now be staring at a blank document.


\end{document}
